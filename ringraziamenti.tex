\section*{Ringraziamenti}
    \textit{Desidero esprimere la mia più profonda gratitudine a tutte le persone che mi hanno sostenuto e accompagnato in questo percorso. Da bravo ingegnere posso solo che ringraziare con un elenco puntato quindi...
    \newline
    \newline
    Un grazie alla mia \textbf{famiglia} per essermi stata accanto in questi anni nonostante il mio essere sgorbutico. Grazie \textbf{Papà} per aver creduto in me ad ogni mio passo, grazie per essermi stato accanto e grazie per avermi lasciato fare la mia strada come lo scout che cammina con il suo zaino in spalla. Grazie \textbf{Mamma} per aver subito i miei "non rompere", "oh Mà, che palle", "quanto stressi" ad ogni tuo "tato, come va?", "stati studiando?", "che si dice con gli esami". Grazie per aver creduto in me in ogni mia decisione. Grazie \textbf{Ale} per essere stato uomo senza il tuo fratellone accanto. Grazie per essere stato il motivo di continua e di riscatto. Grazie per essere il vostro orgoglio.
    \newline
    \newline 
    U grazie ai \textbf{Nonni} per avermi supportato e per essermi stati vicini nel completare questo percorso. Grazie alla nonna Maria per aver creduto in me già da piccolo, per avermi accompagnato in ogni traguardo, per avermi supportato. Grazie al nonno Tonino e al nonno Angelo che, anche se di poche parole, hanno sempre creduto in me e hanno saputo che ce l'averi fatta. Grazie alla nonna Mela per aver supportato il mio percorso e per essere orgogliosa delle scelte che ho fatto. Quello che sono diventato è anche grazie a voi.
    \newline
    \newline
    Un grazie al mio  \textbf{Vituccio}, per essere stato il mio compagno di mille avventure e mille disastri, per essere e rimanere per sempre il mio compare, per essermi rimasto accanto nel bene e nel male, per avermi volute bene dal primo momento che ci siamo incontrati.
    \newline
    \newline
    Un grazie a \textbf{Chiara e Cate}, le mie amiche dell'asse Roma-Laterza, le persone con cui ho condiviso frustrazioni, gioie e bevute post-lavoro. Siete arrivate nel pieno della mia adolescenza, mi siete state accanto in tutti i momenti e anche se la distanza ci divide, chi più chi meno, siete e rimanete un punto di riferimento.
    \newline
    \newline
    Un grazie a \textbf{Ieie}, la mia forza, la mia roccia, il mio sostegno nello sconforto e il motivo per continuare. Grazie per avermi sostenuto nelle mie difficoltà, per aver creduto in me, per avermi spronato a dare il massimo e per avermi sostenuto nelle mie cadute. Grazie per aver asciugato le mie lacrime nella delusione, grazie per avermi sopportato nel mio essere testardo, nel mio diventare sgorbutico e nel mio essere il tuo antipatico. Grazie per essermi stata accanto, per essere arrivata nella mia vita e per averla riempita di emozioni. Grazie per aver pianto insieme, per aver fatto mille avventure, semplicemente grazie per riempire e continuare a riempire la mia vita di gioia nonostante le mie poche parole. Sono orgoglioso di te, di ciò che sei ... di ciò che siamo.
    \newline
    \newline
    Un grazie a \textbf{La Zì} per ricordarmi di credere in me e che la vita va affrontata con leggerezza, col sorriso e non pensare ad abbattersi ma credere in quello che si fà. Grazie per avermi battezzato e per ricordarmelo ogni volta che ne hai modo.
    \newline
    \newline
    Un grazie a \textbf{Gigio, Rena, Michael e Anna} per essere entrati nella mia vita come una bomba e per essere diventati parte fondamentale del mio futuro. Grazie per esserci e per aver sostenuto il mio viaggio tortuoso verso questo traguardo.
    \newline
    \newline
    Un grazie a \textbf{Flaminio e Alessia} per essere stati gli amici di mille e una notte. Per aver accompagnato questo percorso tra panzerotti e focacce, per essermi stati accanto in tutto questo percorso.
    \newline
    \newline
    A tutti voi, va il mio più sincero e affettuoso ringraziamento che avete supportato questo mio percorso. Senza di voi, questo progetto non sarebbe stato possibile.
    \newline
    \newline
    Un grazie a te \textbf{Antonio}, per essere stato tenace nel raggiungere questo obiettivo e per non aver mai permesso a nessuno di scoraggiarti o di interferire con tutto. Ricorda, che tutti sono stati importanti ma solo tu sei stato indispensabile.
}
\vfill
    
\begin{center}
    \textit{"Questo progetto mi ha completamente confiscato la vita, tesoro.}
    \newline\textit{Mi ha consumato come solo un lavoro da eroe riesce a fare. }
    \newline\textit{È il mio capolavoro, lo ammetto:}
    \newline\textit{semplice, elegante eppure importante."}
    \newline 
    \newline\fontsize{8}{6}\textit{Edna "E" Mode da "Gli Incredibili"}
\end{center}