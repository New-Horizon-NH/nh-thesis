\section*{Introduzione}
New-Horizon è un progetto con lo scopo unico di rendere più moderna e unificata la gestione dei reparti ospedalieri, dei pazienti, delle strumentazioni e di qualsiasi altra parte che concerne la vita ospedaliera e i processi dei dipendenti e dei pazienti. 

\subsection*{Contesto e Motivazione}
Nel panorama attuale dello sviluppo software, l'architettura a microservizi sta guadagnando sempre più popolarità grazie alla sua capacità di scalabilità, modularità e resilienza. Le aziende di ogni settore stanno adottando i microservizi per migliorare la gestione delle loro applicazioni e garantire una maggiore agilità nello sviluppo e nella manutenzione del software. Rispetto ai sistemi monolitici, i microservizi permettono di ridurre i tempi di rilascio e di ottimizzare l'allocazione delle risorse computazionali.
\newline Questo progetto si propone di analizzare le principali caratteristiche dei microservizi, confrontandoli con le architetture monolitiche tradizionali, e di implementare un sistema basato su microservizi utilizzando tecnologie moderne come Spring Boot e API REST. Inoltre, verranno approfondite le tecniche di deployment e di gestione della comunicazione tra i servizi, con particolare attenzione alla sicurezza e alla scalabilità del sistema.

\subsection*{Obiettivi del progetto}
L'obiettivo principale di questa tesi è sviluppare un'architettura a microservizi per la gestione di un sistema informativo, con particolare attenzione alla gestione di reparti, esami, pazienti, farmaci e strumentazione ospedaliera. Il progetto include la scelta delle tecnologie più adatte, la progettazione dell'architettura e l'implementazione di un API Gateway per l'integrazione dei servizi. Inoltre, verranno analizzate le strategie di autenticazione e autorizzazione per garantire la sicurezza del sistema e verranno proposte soluzioni per il monitoraggio e il logging delle operazioni.